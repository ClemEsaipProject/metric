\documentclass[a4paper,12pt]{article}
\usepackage[utf8]{inputenc}
\usepackage[T1]{fontenc}
\usepackage[french]{babel}
\usepackage{hyperref}
\usepackage{enumitem}
\usepackage{geometry}
\geometry{margin=2.5cm}

\title{Protocole d'Expérience\\
Analyse comparative des plateformes de streaming sécurisées}
\author{Groupe PSIR}
\date{\today}

\begin{document}

\maketitle

\tableofcontents
\newpage

\section{Préparation de l’environnement}

\subsection{Création du dossier de travail et organisation des fichiers}
\begin{enumerate}[label=\alph*)]
    \item Ouvrir un terminal.
    \item Créer un dossier dédié au projet et s'y placer :
    \begin{verbatim}
    mkdir ~/psir_streaming_exp
    cd ~/psir_streaming_exp
    \end{verbatim}
    \item Cloner le repository :
    \begin{verbatim}
    git clone https://github.com/ClemEsaipProject/metric.git
    \end{verbatim}
    \item Placer le \texttt{Makefile} à la racine de \texttt{~/psir_streaming_exp}, au même niveau que le dossier \texttt{metric}.
    \item Vérifier la structure suivante :
    \begin{verbatim}
    ~/psir_streaming_exp/
    ├── Makefile
    └── metric/
        ├── Dockerfile
        ├── supervisord.conf
        └── (autres fichiers)
    \end{verbatim}
\end{enumerate}

\section{Installation et lancement de l’environnement Docker}

\begin{enumerate}[label=\alph*)]
    \item Construire l’image Docker :
    \begin{verbatim}
    make build
    \end{verbatim}
    \item Lancer le conteneur Docker :
    \begin{verbatim}
    make run
    \end{verbatim}
    \item Les ports exposés sont :
    \begin{itemize}
        \item Grafana : \url{http://localhost:3000}
        \item Prometheus : \url{http://localhost:9090}
        \item Node Exporter : \url{http://localhost:9100}
    \end{itemize}
    \item Arrêter et nettoyer :
    \begin{verbatim}
    make stop      # Arrête et supprime le conteneur
    make clean     # Supprime conteneur et image Docker
    make rebuild   # Nettoie, clone et reconstruit tout
    \end{verbatim}
\end{enumerate}

\section{Prérequis : Configuration de Grafana et Prometheus}

\subsection{Vérification des services}
\begin{itemize}
    \item Ouvrir un navigateur :
    \begin{itemize}
        \item Prometheus : \url{http://localhost:9090}
        \item Grafana : \url{http://localhost:3000}
        \item Node Exporter : \url{http://localhost:9100}
    \end{itemize}
\end{itemize}

\subsection{Configuration de Prometheus}
\begin{itemize}
    \item Vérifier que le fichier \texttt{prometheus.yml} contient :
\begin{verbatim}
scrape_configs:
  - job_name: 'prometheus'
    static_configs:
      - targets: ['localhost:9090']
  - job_name: 'node-exporter'
    static_configs:
      - targets: ['localhost:9100']
\end{verbatim}
    \item Adapter les cibles si besoin (multi-conteneur Docker).
\end{itemize}

\subsection{Ajout de Prometheus dans Grafana}
\begin{enumerate}[label=\alph*)]
    \item Se connecter à Grafana (admin/admin).
    \item Aller dans \textbf{Configuration > Data Sources}.
    \item Ajouter une source de données \textbf{Prometheus}.
    \item Renseigner l’URL : \texttt{http://localhost:9090} ou \texttt{http://prometheus:9090}.
    \item Cliquer sur \textbf{Save \& Test}.
\end{enumerate}

\subsection{Importation d’un dashboard Grafana}
\begin{itemize}
    \item Aller dans \textbf{Dashboards > Import}.
    \item Utiliser l’ID \texttt{1860} (Node Exporter Full) ou créer un dashboard personnalisé.
    \item Sélectionner la source de données Prometheus.
\end{itemize}

\section{Protocole de mesure : étapes détaillées}

\subsection{Préparation de la capture réseau}
\textbf{Pourquoi ?} Capturer tout le trafic réseau du test, y compris l’établissement de la connexion.

\textbf{Comment faire ?}
\begin{enumerate}[label=\alph*)]
    \item Accéder au conteneur Docker :
    \begin{verbatim}
    docker exec -it outils-mesure bash
    \end{verbatim}
    \item Identifier l’interface réseau (souvent \texttt{eth0}) :
    \begin{verbatim}
    ip addr
    \end{verbatim}
    \item Lancer la capture avec tshark (remplacer \texttt{<nom\_fichier>} par un nom explicite) :
    \begin{verbatim}
    tshark -i eth0 -w /data/<nom_fichier>.pcap
    \end{verbatim}
    \item Laisser la capture tourner, \textbf{ne pas envoyer le flux avant cette étape}.
\end{enumerate}

\subsection{Envoi du flux vers la plateforme cible}
\textbf{Pourquoi ?} Tester en conditions réelles la gestion du chiffrement, de l’anonymisation, la latence et la sécurité.

\textbf{Comment faire ?}
\begin{enumerate}[label=\alph*)]
    \item Récupérer l’URL et la clé de stream de la plateforme.
    \item Générer et envoyer le flux vidéo de test (exemple 4K) :
    \begin{verbatim}
    ffmpeg -re -f lavfi -i testsrc=size=3840x2160:rate=30 -c:v libx264 -preset veryfast -f flv rtmp://URL_SERVEUR/cle_stream
    \end{verbatim}
    \item Vérifier sur la plateforme que le flux est bien reçu.
\end{enumerate}

\subsection{Arrêt de la capture réseau}
\textbf{Pourquoi ?} Obtenir un fichier pcap propre, limité à la session étudiée.

\textbf{Comment faire ?}
\begin{itemize}
    \item Arrêter la capture avec \texttt{Ctrl+C}.
    \item Le fichier \texttt{/data/<nom\_fichier>.pcap} contient tous les paquets du test.
\end{itemize}

\subsection{Analyse du trafic réseau}
\textbf{Pourquoi ?} Mesurer latence, débit, sécurité du transport.

\textbf{Comment faire ?}
\begin{enumerate}[label=\alph*)]
    \item Ouvrir le fichier pcap dans Wireshark.
    \item Utiliser les filtres pour isoler le trafic d’intérêt (\texttt{tcp}, \texttt{udp}, \texttt{ip.addr==...}).
    \item Mesurer : latence (timestamps), débit (Statistiques $>$ Conversations), sécurité (données sensibles en clair).
\end{enumerate}

\subsection{Mesure de la qualité vidéo}
\textbf{Pourquoi ?} Quantifier la dégradation via PSNR, SSIM, VMAF.

\textbf{Comment faire ?}
\begin{enumerate}[label=\alph*)]
    \item Récupérer la vidéo traitée par la plateforme.
    \item Comparer à l’originale :
    \begin{verbatim}
    ffmpeg -i video_recue.mp4 -i video_originale.mkv -lavfi "libvmaf=psnr=1:ssim=1:log_path=metrics.json" -f null -
    \end{verbatim}
    \item Consulter les scores dans \texttt{metrics.json}.
\end{enumerate}

\subsection{Surveillance des ressources système}
\textbf{Pourquoi ?} Évaluer l’impact sur CPU, RAM, stabilité.

\textbf{Comment faire ?}
\begin{itemize}
    \item Accéder à Grafana (\url{http://localhost:3000}).
    \item Consulter les dashboards (Node Exporter Full).
\end{itemize}

\subsection{Tests de sécurité et d’anonymisation}
\textbf{Pourquoi ?} Vérifier robustesse du chiffrement, résistance MITM, anonymisation.

\textbf{Comment faire ?}
\begin{enumerate}[label=\alph*)]
    \item Lancer OWASP ZAP dans le conteneur :
    \begin{verbatim}
/usr/local/bin/zaproxy &
    \end{verbatim}
    \item Configurer un scan sur l’URL du flux ou du service cible.
    \item Analyser les alertes (failles TLS, données sensibles non chiffrées, etc.).
    \item Vérifier visuellement ou via script Python/OpenCV que les zones sensibles sont floutées.
\end{enumerate}

\section{Organisation des données et bonnes pratiques}
\begin{itemize}
    \item Créer un dossier par plateforme et par scénario pour stocker captures, vidéos, métriques et logs.
    \item Noter systématiquement : date, heure, plateforme, configuration réseau, scénario, paramètres de test.
    \item Centraliser les résultats dans un tableau Excel ou Sheets pour l’analyse comparative.
\end{itemize}

\section{Récapitulatif du workflow}

\begin{tabular}{|p{4.5cm}|p{9cm}|}
\hline
\textbf{Étape} & \textbf{Commande ou action clé} \\
\hline
Création du dossier projet & \texttt{mkdir ~/psir_streaming_exp \&\& cd ~/psir_streaming_exp} \\
Clonage du repo & \texttt{git clone https://github.com/ClemEsaipProject/metric.git} \\
Placement du Makefile & Placez-le à la racine, à côté de \texttt{metric} \\
Build de l’image & \texttt{make build} \\
Lancement du conteneur & \texttt{make run} \\
Configuration Grafana/Prom & Voir section dédiée \\
Préparation capture réseau & \texttt{docker exec -it outils-mesure bash} puis \texttt{tshark -i eth0 -w /data/capture.pcap} \\
Génération/envoi du flux & \texttt{ffmpeg ... -f flv rtmp://URL/cle_stream} \\
Arrêt de la capture & \texttt{Ctrl+C} dans le terminal tshark \\
Analyse qualité vidéo & \texttt{ffmpeg -i video\_recue.mp4 -i video\_originale.mkv -lavfi "libvmaf=..." -f null -} \\
Monitoring ressources & Accès Grafana \url{http://localhost:3000} \\
Sécurité/anonymisation & \texttt{/usr/local/bin/zaproxy \&} puis scan sur le flux \\
Arrêt/nettoyage & \texttt{make stop} ou \texttt{make clean} \\
\hline
\end{tabular}

\vspace{1em}

\section{Suggestions d’amélioration}
\begin{itemize}
    \item Documenter chaque test dans un carnet de laboratoire pour la traçabilité.
    \item Automatiser la collecte et l’archivage des résultats via des scripts shell ou Python.
    \item Ajouter des exemples de scripts d’analyse pour l’exploitation scientifique des données.
    \item Inclure un fichier \texttt{README.md} rappelant la structure et les commandes principales.
    \item Vérifier régulièrement les versions des outils pour garantir la reproductibilité.
\end{itemize}

\end{document}
